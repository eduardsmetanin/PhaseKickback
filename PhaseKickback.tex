\documentclass[12pt,twoside,a4paper]{article}
\author{Eduard Smetanin}
\title{\vspace{-2.0cm}Phase Kickback}
\usepackage{a4wide}
\usepackage{amsmath}
\usepackage{qcircuit}
\usepackage{braket}
\usepackage{chemarrow}
\usepackage{setspace}
\onehalfspacing
\usepackage{hyperref}
\begin{document}
\maketitle

Phase kickback is used in many quantum algorithms, for instance Deutsch's, Simon’s, Shor’s, Grover's and the phase estimation algorithms.

Let's consider quantum circuit
\begin{equation*}
        \Qcircuit @C=1em @R=1em {
                \lstick{\ket{+}}
                & \qw & \ctrl{1} & \qw & \meter \\
                \lstick{\ket{\psi}} & \qw & \gate{U} & \qw & \qw \\
                & \uparrow & & \uparrow \\
                & A & & B
        }
\end{equation*}
where $U$ is an arbitrary gate; state $\ket{+} \equiv \ket{+x} \equiv H \ket{0} = \frac{1}{\sqrt{2}} (\ket{0} + \ket{1})$, where $H$ is Hadamard gate.

At point $A$ the system is in state
$\ket{+} \ket{\psi} = \frac{\ket{0} + \ket{1}}{\sqrt{2}} \ket{\psi}
= \frac{\ket{0} \ket{\psi} + \ket{1} \ket{\psi}}{\sqrt{2}}$.

At point $B$ the system is in state
$\frac{\ket{0} \ket{\psi} + \ket{1} U \ket{\psi}}{\sqrt{2}}$.
$U$ gets applied only to $\ket{\psi}$ when the target qubit is $\ket{1}$ because $\begin{array}{c} \Qcircuit @C=.25em @R=.3em {& \ctrl{1} & \qw \\ & \gate{U} & \qw} \end{array}$ is a controlled operation that triggers when control qubit is $\ket{1}$.
Generally speaking, controlled operations can be triggered not only on $\ket{1}$, but also on $\ket{0}$ or any function of the control qubit(s).
We will discuss this later.

In general case, after $U$ got applied to $\ket{\psi}$ when control qubit is $\ket{1}$, it's impossible to factor out original $\ket{\psi}$ from state $\frac{\ket{0} \ket{\psi} + \ket{1} U \ket{\psi}}{\sqrt{2}}$.

However, if $\ket{\psi}$ happened to be an eigenvector of $U$ (with some eigenvalue $e^{i\phi}$), something interesting happens instead.
In this case $U \ket{\psi} = e^{i\phi} \ket{\psi}$.
At point $B$ the system is in state
$\frac{\ket{0} \ket{\psi} + \ket{1} e^{i\phi} \ket{\psi}}{\sqrt{2}}
= \frac{\ket{0} + e^{i\phi} \ket{1}}{\sqrt{2}}  \ket{\psi}$.
Since $e^{i\phi}$ is just a number, we can factor out $\ket{\psi}$ this time.
Because we can factor it out and because there is only one way to split composite state into individual states (up to a global phase), $e^{i\phi}$ will be with the first qubit when factored into individual states.
Of course, if the system were entangled, we wouldn't be able to separate state into individual states, but this is not the case this time.
So, in case when $\ket{\psi}$ is an eigenvector of $U$ the target qubit doesn't change, but control qubit changes in such a way that one of its amplitudes gets multiplied by the eigenvalue of the eigenvector.
This phenomenon is called phase kickback.

The requirements for phase kickback to occur are:
\begin{itemize}
        \item \emph{The control qubit(s) must be in a superposition.} \\
        This is because otherwise applying controlled operation will just change global phase:
\begin{equation*}
        \ket{1} \ket{\psi} \autorightarrow{\Qcircuit @C=.25em @R=.3em {& \ctrl{1} & \qw \\ & \gate{U} & \qw}}{} \ket{1} U \ket{\psi} = \ket{1} e^{i\phi} \ket{\psi} \cong \ket{1} \ket{\psi}
\end{equation*}
        Global phase has no physical meaning, so it wouldn't change the physical state.
        \item \emph{$\ket{\psi}$ must be an eigenvector of operator $U$.} \\
        We already discussed this.
        \item \emph{Arbitrary operator $U$ must be applied in a controlled way.} \\
        Often $U$ gets triggered on $\ket{0}$ or $\ket{1}$, but as we mentioned, it can be any function of the control qubit.
        The reason why $U$ has to be applied in a controlled way is if we apply $U$ unconditionally, we would just change global phase of a state.
        At point $B$ state would be
        \begin{equation*}
                U \frac{\ket{0} + \ket{1}}{\sqrt{2}} \ket{\psi}
                = e^{i\phi} \frac{\ket{0} + \ket{1}}{\sqrt{2}} \ket{\psi}
                \cong \frac{\ket{0} + \ket{1}}{\sqrt{2}} \ket{\psi}
        \end{equation*}
\end{itemize}

Let's take a look at an example of how phase kickback works.
We will use a simple search algorithm described in \cite{watrous5}.

Suppose we are given a function that inputs two qubits and outputs one qubit along with a promise it's one of the four functions:
\begin{center}
        \begin{tabular}[t]{|c|c|}
                \hline
                \multicolumn{2}{|c|}{$f_{00}$} \\
                \hline
                input & output \\
                \hline
                00 & 1 \\
                01 & 0 \\
                10 & 0 \\
                11 & 0 \\
                \hline
        \end{tabular}
        \begin{tabular}[t]{|c|c|}
                \hline
                \multicolumn{2}{|c|}{$f_{01}$} \\
                \hline
                input & output \\
                \hline
                00 & 0 \\
                01 & 1 \\
                10 & 0 \\
                11 & 0 \\
                \hline
        \end{tabular}
        \begin{tabular}[t]{|c|c|}
                \hline
                \multicolumn{2}{|c|}{$f_{10}$} \\
                \hline
                input & output \\
                \hline
                00 & 0 \\
                01 & 0 \\
                10 & 1 \\
                11 & 0 \\
                \hline
        \end{tabular}
        \begin{tabular}[t]{|c|c|}
                \hline
                \multicolumn{2}{|c|}{$f_{11}$} \\
                \hline
                input & output \\
                \hline
                00 & 0 \\
                01 & 0 \\
                10 & 0 \\
                11 & 1 \\
                \hline
        \end{tabular}
\end{center}
Note, the functions outputs $1$ for one input and $0$ for all others.

The goal is to determine which function is implemented by the circuit without looking inside of $U_f$ (see below).
The rest of the circuit is only allowed to evaluate the function.
To determine which function is implemented is the same as to find the input on which the function outputs $1$.

This circuit solves the problem:
\begin{equation*}
        \Qcircuit @C=1em @R=1em {
                \lstick{\ket{+}} & \qw & \multigate{2}{U_f} & \qw & \multigate{1}{M} & \qw & \meter \\
                \lstick{\ket{+}} & \qw & \ghost{U_f} & \qw & \ghost{U_f} & \qw & \meter \\
                \lstick{\ket{-}} & \qw & \ghost{U_f} & \qw & \qw & \qw & \qw & trash \\
                & \uparrow & & \uparrow \\
                & A & & B
        }
\end{equation*}
$M$ is a circuit that converts output of $U_f$ to allow find the solution by measuring.
We are not going to discuss how it's implemented here since this is not directly related to phase kickback.

State $\ket{+} \equiv \ket{+x} \equiv H \ket{0} = \frac{1}{\sqrt{2}} (\ket{0} + \ket{1})$ and $\ket{-} \equiv \ket{-x} \equiv H \ket{1} = \frac{1}{\sqrt{2}} (\ket{0} - \ket{1})$, where $H$ is Hadamard gate.

$U_f$ is implemented in a way that is very common for a lot of quantum algorithms:
\begin{equation*}
        \Qcircuit @C=1em @R=1em {
                \lstick{\ket{a}} & \qw & \multigate{2}{U_f} & \qw & \qw & \ket{a} \\
                \lstick{\ket{b}} & \qw & \ghost{U_f} & \qw & \qw & \ket{b} \\
                \lstick{\ket{c}} & \qw & \ghost{U_f} & \qw & \qw & & & \ket{c \oplus f(a, b)}
        }
\end{equation*}
where $a$ and $b$ are control qubits and $c$ is target qubit that gets flipped when $f(a, b)$ outputs $1$.
So, $U_f$ is a $control {\text -} f {\text -} NOT$ gate.
Notice that state of the target qubit $\ket{-}$ is an eigenvector of $NOT$ gate.

At point $A$ the system is in state
$\ket{+} \ket{+} \ket{-} = \frac{1}{2} (\ket{00} + \ket{01} + \ket{10} + \ket{11}) \ket{-}
= \frac{1}{2} (\ket{00} \ket{-} + \ket{01} \ket{-} + \ket{10} \ket{-} + \ket{11} \ket{-})$.
We keep the last qubit as short $\ket{-}$ instead of $\frac{1}{\sqrt{2}} (\ket{0} - \ket{1})$ because it won't change.

At point $B$, when function $f = f_{00}$, the system is in state
\begin{equation*}
        \begin{split}
                \frac{1}{2} (\ket{00} U_{f_{00}} \ket{-} + \ket{01} \ket{-} + \ket{10} \ket{-} + \ket{11} \ket{-}) \\
                = \frac{1}{2} (\ket{00} (-1) \ket{-} + \ket{01} \ket{-} + \ket{10} \ket{-} + \ket{11} \ket{-}) \\
                = \frac{1}{2} (- \ket{00} + \ket{01} + \ket{10} + \ket{11}) \ket{-}.
        \end{split}
\end{equation*}
Notice how phase of $\ket{00}$ gets flagged by multiplying it by eigenvalue of $\ket{-}$ which is $-1$.
Later on this flag will be converted into classical information that will allow us to receive the result of the computation.

Similarly, at point $B$, when function $f = f_{01}$, the system is in state
\begin{equation*}
        \begin{split}
                \frac{1}{2} (\ket{00} \ket{-} + \ket{01} U_{f_{01}} \ket{-} + \ket{10} \ket{-} + \ket{11} \ket{-}) \\
                = \frac{1}{2} (\ket{00} \ket{-} + \ket{01} (-1) \ket{-} + \ket{10} \ket{-} + \ket{11} \ket{-}) \\
                = \frac{1}{2} (\ket{00} - \ket{01} + \ket{10} + \ket{11}) \ket{-}.
        \end{split}
\end{equation*}
So, for every of the four functions $f$, the corresponding phase gets flagged to be later converted to the solution of the problem.

\cite{watrous}, \cite{gokhale} and \cite{zhou} contain additional information about phase kickback.

\begin{thebibliography}{99}
        \bibitem{watrous5} \hypertarget{watrous5}{} John Watrous - CPSC 519/619: Quantum Computation
        - Lecture 5 - 2006 \\
        \url{https://cs.uwaterloo.ca/~watrous/LectureNotes/CPSC519.Winter2006/all.pdf#page=28}

        \bibitem{watrous} \hypertarget{watrous}{} John Watrous - CPSC 519/619: Quantum Computation - 2006 \\
        \url{https://cs.uwaterloo.ca/~watrous/LectureNotes/CPSC519.Winter2006/all.pdf}

        \bibitem{gokhale} \hypertarget{gokhale}{} Pranav Gokhale - What is phase kickback and how does it occur? \\
        \url{https://www.quora.com/What-is-phase-kickback-and-how-does-it-occur}

        \bibitem{zhou} \hypertarget{zhou}{} Dong Zhou - Phase kickback \\
        \url{https://nosarthur.github.io/quantum%20information%20and%20computation/2018/01/26/kickback.html}
\end{thebibliography}

\end{document}
